% ============================================================================
%  Formation Engine — §0: Formal Identity–State Decomposition Framework
%  Printer-ready LaTeX.  Compile with pdflatex (twice for refs).
% ============================================================================
\documentclass[11pt, a4paper, oneside]{article}

% ── Geometry ────────────────────────────────────────────────────────────────
\usepackage[
  top=1.0in, bottom=1.0in, left=1.15in, right=1.15in,
  headheight=14pt
]{geometry}

% ── Fonts / encoding ───────────────────────────────────────────────────────
\usepackage[T1]{fontenc}
\usepackage{lmodern}
\usepackage{microtype}

% ── Math ───────────────────────────────────────────────────────────────────
\usepackage{amsmath, amssymb, amsthm, mathtools}
\usepackage{bm}

% ── Tables ─────────────────────────────────────────────────────────────────
\usepackage{booktabs}
\usepackage{array}
\usepackage{tabularx}

% ── Lists ──────────────────────────────────────────────────────────────────
\usepackage{enumitem}
\setlist{nosep, leftmargin=1.5em}

% ── Headers / footers ─────────────────────────────────────────────────────
\usepackage{fancyhdr}
\pagestyle{fancy}
\fancyhf{}
\fancyhead[L]{\small\textit{Formation Engine Methodology}}
\fancyhead[R]{\small\textit{§0 — Identity–State Decomposition}}
\fancyfoot[C]{\thepage}
\renewcommand{\headrulewidth}{0.4pt}

% ── Section formatting ─────────────────────────────────────────────────────
\usepackage{titlesec}
\titleformat{\section}
  {\Large\bfseries}
  {§\thesection}{1em}{}
\titleformat{\subsection}
  {\large\bfseries}
  {\thesubsection}{1em}{}
\titleformat{\subsubsection}
  {\normalsize\bfseries}
  {\thesubsubsection}{1em}{}

\setcounter{section}{-1}  % first \section produces §0

% ── Hyperlinks ─────────────────────────────────────────────────────────────
\usepackage[
  colorlinks=true, linkcolor=black, citecolor=black, urlcolor=blue
]{hyperref}

% ── Theorem-like environments ──────────────────────────────────────────────
\theoremstyle{definition}
\newtheorem{postulate}{Postulate}[section]
\newtheorem{defn}[postulate]{Definition}
\theoremstyle{remark}
\newtheorem{remark}[postulate]{Remark}

% ── Convenience macros ─────────────────────────────────────────────────────
\newcommand{\R}{\mathbb{R}}
\newcommand{\N}{\mathbb{N}}
\newcommand{\Z}{\mathbb{Z}}
\newcommand{\Bset}{\mathbb{B}}
\newcommand{\Kcal}{\text{kcal}}
\newcommand{\mol}{\text{mol}}
\newcommand{\boldI}{\mathbf{I}}
\newcommand{\boldG}{\mathbf{G}}
\newcommand{\boldC}{\mathbf{C}}
\newcommand{\boldK}{\mathbf{K}}
\newcommand{\boldL}{\mathbf{L}}
\newcommand{\boldS}{\mathbf{S}}
\DeclareMathOperator{\Var}{Var}
\DeclareMathOperator*{\argmax}{arg\,max}

% ── Document ───────────────────────────────────────────────────────────────
\begin{document}

% ── Title block ────────────────────────────────────────────────────────────
\begin{center}
  {\LARGE\bfseries Formation Engine}\\[0.3em]
  {\LARGE\bfseries Canonical Simulation Methodology}\\[1.2em]
  {\large §0 — Formal Identity–State Decomposition Framework}\\[1.5em]
  {\normalsize
    \textit{A deterministic classical-mechanical framework for the prediction
    of molecular and crystalline structure from elemental identity and
    thermodynamic boundary conditions, without recourse to empirical
    molecular data.}}\\[1.5em]
  {\small Version 0.1 — First Principles Draft}
\end{center}

\vspace{1em}
\hrule
\vspace{2em}

% ════════════════════════════════════════════════════════════════════════════
\section{Formal Identity–State Decomposition Framework}
% ════════════════════════════════════════════════════════════════════════════

% ────────────────────────────────────────────────────────────────────────────
\subsection{Identity}
\label{sec:identity-vector}
% ────────────────────────────────────────────────────────────────────────────

Define a time-invariant identity vector for particle~$i$:
%
\begin{equation}
\boldI_i \in \R^{6}, \qquad
\boldI_i =
\begin{bmatrix}
  Z_i \\ A_i \\ Q_i \\ \Sigma_i \\ \Lambda_i \\ \Theta_i
\end{bmatrix}
\label{eq:identity-vector}
\end{equation}
%
with components defined as follows.

% ·········································································
\subsubsection{Nuclear Identity \texorpdfstring{$Z_i$}{Zi}}
\label{sec:nuclear-identity}

\[
  Z_i \in \N^{+}
\]

Atomic number.  Immutable under all non-nuclear transformations.
Serves as the identity anchor for all downstream mappings.

% ·········································································
\subsubsection{Mass Identity \texorpdfstring{$A_i$}{Ai}}
\label{sec:mass-identity}

\[
  A_i \in \R^{+}
\]

Represents either:
\begin{itemize}
  \item exact mass number (isotopic resolution), or
  \item effective mass bucket under coarse-graining.
\end{itemize}

\noindent Constraint:
\[
  A_i^{(\text{fast})} \approx \bar{A}(Z_i),
  \qquad
  A_i^{(\text{audit})} = A_i^{\,\text{exact}}
\]

% ·········································································
\subsubsection{Effective Charge Participation
              \texorpdfstring{$Q_i$}{Qi}}
\label{sec:charge-participation}

\[
  Q_i \in \R
\]

Not an oxidation state.  Instead, define $Q_i$ as a participation
coefficient modulating interaction strength:
%
\begin{equation}
  Q_i = \frac{1}{|\mathcal{N}_i|}
        \sum_{j \in \mathcal{N}_i} \chi_{ij}
  \label{eq:charge-participation}
\end{equation}
%
where $\chi_{ij}$ is an interaction asymmetry measure (electron
donation/withdrawal bias) and $\mathcal{N}_i$ is the local
neighbourhood.

This allows:
\begin{itemize}
  \item fractional values,
  \item environment dependence,
  \item smooth behaviour under coarse-graining.
\end{itemize}

% ·········································································
\subsubsection{Structural Role Signature
              \texorpdfstring{$\Sigma_i$}{Sigma-i}}
\label{sec:structural-role}

\[
  \Sigma_i \in \{0,\,1,\,2,\,3,\,4\}
\]

Discrete structural prior encoding dominant bonding topology:
%
\[
  \Sigma_i =
  \begin{cases}
    0 & \text{inert / closed shell}    \\
    1 & \text{ionic-dominant}           \\
    2 & \text{directional covalent}     \\
    3 & \text{metallic / delocalized}   \\
    4 & \text{mixed / transitional}
  \end{cases}
\]
%
This parameter biases search and interaction kernels without
explicit bond constraints.

% ·········································································
\subsubsection{Stability Class
              \texorpdfstring{$\Lambda_i$}{Lambda-i}}
\label{sec:stability-class}

\[
  \Lambda_i \in \{0,\,1,\,2,\,3\}
\]

Defines statistical persistence under thermal and configurational
perturbation:
%
\begin{equation}
  \Lambda_i = \argmax_k \;
    P_i\!\bigl(\text{survival} \mid \Delta E,\, T,\, t\bigr)
  \label{eq:stability-class}
\end{equation}

Interpretation:
%
\begin{center}
\begin{tabular}{cl}
  \toprule
  $\Lambda$ & Meaning \\
  \midrule
  0 & transient \\
  1 & metastable \\
  2 & ambient-stable \\
  3 & bulk-lattice candidate \\
  \bottomrule
\end{tabular}
\end{center}

% ·········································································
\subsubsection{Provenance Memory
              \texorpdfstring{$\Theta_i$}{Theta-i}}
\label{sec:provenance-memory}

\[
  \Theta_i \in \Bset^{n}, \qquad n \in [8,\,16]
\]

Binary provenance hash encoding transformation lineage:
%
\begin{equation}
  \Theta_i = \mathcal{H}\!\bigl(
    \text{origin},\;
    \text{generation},\;
    \text{relaxation depth}
  \bigr)
  \label{eq:provenance-hash}
\end{equation}

Used for:
\begin{itemize}
  \item audit trails,
  \item anomaly detection,
  \item irreversible-history awareness.
\end{itemize}


% ════════════════════════════════════════════════════════════════════════════
%  PRIMED EXTENSION: Canonical Particle Container Postulate
% ════════════════════════════════════════════════════════════════════════════

\bigskip
\subsection*{0.1.3$'$\quad Canonical Particle Container Postulate (CPCP)}
\addcontentsline{toc}{subsection}%
  {0.1.3$'$ Canonical Particle Container Postulate (CPCP)}
\label{sec:cpcp}

% ·········································································
\subsubsection*{0.1.3$'$.1\quad Postulate Statement}
\addcontentsline{toc}{subsubsection}{0.1.3$'$.1 Postulate Statement}
\label{sec:cpcp-postulate}

\begin{postulate}[Canonical Particle Container]
Any time-independent identical atomic or subatomic particle species
admits an idealised canonical container representation:
%
\begin{equation}
  \mathcal{C}_i =
  \bigl(
    \boldI_i,\;
    \boldG_i,\;
    \boldC_i^{\,\mathrm{comp}}
  \bigr)
  \label{eq:container}
\end{equation}
%
where:
\begin{itemize}
  \item $\boldI_i$ = Identity vector (§\ref{sec:identity-vector}),
  \item $\boldG_i$ = Fundamental internal quantum descriptor set,
  \item $\boldC_i^{\,\mathrm{comp}}$ = Computational annotation payload
        (non-physical but simulation-critical).
\end{itemize}
\end{postulate}

% ·········································································
\subsubsection*{0.1.3$'$.2\quad Fundamental Quantum Descriptor Bundle
               \texorpdfstring{$\boldG_i$}{Gi}}
\addcontentsline{toc}{subsubsection}%
  {0.1.3$'$.2 Fundamental Quantum Descriptor Bundle}
\label{sec:quantum-bundle}

For any identical species class:
%
\begin{equation}
  \boldG_i =
  \begin{bmatrix}
    s_i \\ c_i \\ f_i \\ \kappa_i \\ \eta_i
  \end{bmatrix}
  \label{eq:quantum-bundle}
\end{equation}

\paragraph{Spin \texorpdfstring{$s_i$}{si}.}

\[
  s_i \in \tfrac{1}{2}\,\Z
\]

Intrinsic angular momentum.
Defines statistics class:
%
\[
  s_i \in \Z \;\Rightarrow\; \text{Bosonic regime},
  \qquad
  s_i \in \tfrac{1}{2} + \Z \;\Rightarrow\; \text{Fermionic regime}
\]

\paragraph{Colour charge \texorpdfstring{$c_i$}{ci}.}

For quark-level modelling:
\[
  c_i \in \{r,\,g,\,b,\,\bar{r},\,\bar{g},\,\bar{b}\}
\]
or abstracted as $c_i \in \Z_3$.
Under coarse projection: $\langle c_i \rangle \to 0$
(enforced colour neutrality constraint).

\paragraph{Flavour index \texorpdfstring{$f_i$}{fi}.}

Discrete particle family label:
\[
  f_i \in \N_{\text{flavour classes}}
\]
For atoms, this collapses to nuclear composition mapping via
$(Z_i, A_i)$.
For subatomic simulations, it remains explicit.

\paragraph{Internal projection coupling
          \texorpdfstring{$\kappa_i$}{kappa-i}.}

Represents coupling between core particle and embedded field
carriers:
%
\begin{equation}
  \kappa_i =
  \begin{bmatrix}
    \kappa^{(g)} \\[3pt]
    \kappa^{(em)} \\[3pt]
    \kappa^{(w)}
  \end{bmatrix}
  \label{eq:kappa}
\end{equation}
%
mapping to embedded co-particle field contributions (graviton /
dimagneton / magneton-style internal carriers).

\paragraph{Projection information density
          \texorpdfstring{$\eta_i$}{eta-i}.}

Minimal information content proxy:
%
\begin{equation}
  \eta_i = \frac{I_{\text{struct}}}{V_{\text{proj}}}
  \label{eq:info-density}
\end{equation}
%
Used for:
\begin{itemize}
  \item collapse behaviour heuristics,
  \item coarse-graining fidelity loss tracking,
  \item MIT alignment scoring.
\end{itemize}

% ·········································································
\subsubsection*{0.1.3$'$.3\quad Canonical Container Geometry
               Approximation}
\addcontentsline{toc}{subsubsection}%
  {0.1.3$'$.3 Canonical Container Geometry Approximation}
\label{sec:container-geometry}

For computational purposes, define an effective interaction envelope:
%
\begin{equation}
  \Omega_i =
  \bigl(
    R_i,\;
    \mathcal{T}_i(\theta,\phi),\;
    \rho_i(r)
  \bigr)
  \label{eq:envelope}
\end{equation}

\paragraph{Effective radius.}

\begin{equation}
  R_i = \alpha_Z \, Z_i^{\,1/3} + \alpha_Q \, |Q_i|
  \label{eq:effective-radius}
\end{equation}

\paragraph{Angular texture.}

\begin{equation}
  \mathcal{T}_i(\theta,\phi) =
  \sum_{\ell=0}^{L}
  \sum_{m=-\ell}^{\ell}
  a_{\ell m}\, Y_{\ell m}(\theta,\phi)
  \label{eq:angular-texture}
\end{equation}

\paragraph{Radial density envelope.}

\begin{equation}
  \rho_i(r) = \rho_0 \, e^{-r/\lambda_i}
  \label{eq:radial-density}
\end{equation}

This allows identical species to share container topology while
differing only in parameterisation.


% ════════════════════════════════════════════════════════════════════════════
%  PRIMED EXTENSION: Computational Annotation Extension Layer
% ════════════════════════════════════════════════════════════════════════════

\bigskip
\subsection*{0.1.4$'$\quad Computational Annotation Extension Layer}
\addcontentsline{toc}{subsection}%
  {0.1.4$'$ Computational Annotation Extension Layer}
\label{sec:comp-annotation}

These attributes are not fundamental physics.
They exist because compute is finite and because unstructured
metadata degrades into noise.

Define:
%
\begin{equation}
  \boldC_i^{\,\mathrm{comp}} =
  \begin{bmatrix}
    \Gamma_i \\ \Xi_i \\ \Pi_i \\ \Upsilon_i
  \end{bmatrix}
  \label{eq:comp-annotation}
\end{equation}

\paragraph{Complexity class tag \texorpdfstring{$\Gamma_i$}{Gamma-i}.}

\[
  \Gamma_i \in \{0,\,1,\,2,\,3,\,4\}
\]

Used for adaptive solver selection:
%
\begin{center}
\begin{tabular}{cl}
  \toprule
  Value & Meaning \\
  \midrule
  0 & inert, low interaction \\
  1 & pairwise dominant \\
  2 & directional bonding \\
  3 & collective electron behaviour \\
  4 & multi-scale coupled \\
  \bottomrule
\end{tabular}
\end{center}

\paragraph{Interaction kernel selector
          \texorpdfstring{$\Xi_i$}{Xi-i}.}

Maps particle to preferred interaction kernel family:
\[
  \Xi_i \in
  \{\,\text{LJ},\;
     \text{Coulomb},\;
     \text{Screened},\;
     \text{EAM-like},\;
     \text{Hybrid}\,\}
\]
Allows hot-swapping physics models without identity rewrite.

\paragraph{Solver priority weight
          \texorpdfstring{$\Pi_i$}{Pi-i}.}

\[
  \Pi_i \in [0,\,1]
\]

Used in adaptive timestep / refinement scheduling:
%
\begin{equation}
  \Delta t_i = \frac{\Delta t_{\max}}{1 + \beta\,\Pi_i}
  \label{eq:adaptive-dt}
\end{equation}

\paragraph{Annotation confidence
          \texorpdfstring{$\Upsilon_i$}{Upsilon-i}.}

Tracks reliability of assigned metadata:
\begin{equation}
  \Upsilon_i = 1 - P(\text{misclassification})
  \label{eq:annotation-confidence}
\end{equation}
Critical for self-audit subsystems.

% ·········································································
\subsubsection*{0.1.4$'$.5\quad Container Invariance Condition}
\addcontentsline{toc}{subsubsection}%
  {0.1.4$'$.5 Container Invariance Condition}
\label{sec:container-invariance}

For identical species:
%
\begin{equation}
  \mathcal{C}_i \equiv \mathcal{C}_j
  \quad \text{iff} \quad
  \boldI_i = \boldI_j
  \;\land\;
  \boldG_i = \boldG_j
  \label{eq:container-invariance}
\end{equation}

But computational layers may differ:
\[
  \boldC_i^{\,\mathrm{comp}} \neq \boldC_j^{\,\mathrm{comp}}
\]

This is deliberate and necessary for performance scaling.

% ·········································································
\subsubsection*{0.1.4$'$.6\quad Practical Consequence}
\addcontentsline{toc}{subsubsection}%
  {0.1.4$'$.6 Practical Consequence}
\label{sec:container-consequence}

\noindent\textbf{Physics layer.}
Identical particle $=$ identical fundamental container.

\medskip
\noindent\textbf{Compute layer.}
Identical particle can be simulated differently depending on context.

\medskip
\noindent
That is exactly how you triple effective data density without
multiplying particle count or neighbour graph size.


% ════════════════════════════════════════════════════════════════════════════
\subsection{Cells}
\label{sec:cell-container}
% ════════════════════════════════════════════════════════════════════════════

% ·········································································
\subsubsection{Postulate}
\label{sec:cell-postulate}

A simulation ``cell'' (molecular box, periodic crystal cell, or
derived supercell) is represented as a minimal container:
%
\begin{equation}
  \mathcal{K} =
  \bigl(
    \boldK^{\mathrm{id}},\;
    \boldK^{\mathrm{geom}},\;
    \boldK^{\mathrm{comp}}
  \bigr)
  \label{eq:cell-container}
\end{equation}
%
where:
\begin{itemize}
  \item $\boldK^{\mathrm{id}}$ is immutable lineage / provenance,
  \item $\boldK^{\mathrm{geom}}$ is the geometric definition of the cell,
  \item $\boldK^{\mathrm{comp}}$ is the minimal computational annotation
        payload (caches are explicitly \emph{not} part of the container).
\end{itemize}

% ·········································································
\subsubsection{Lineage \texorpdfstring{$\boldK^{\mathrm{id}}$}{Kid}}
\label{sec:cell-identity}

\begin{equation}
  \boldK^{\mathrm{id}} =
  \begin{bmatrix}
    \Theta_{\mathcal{K}} \\[3pt]
    \Pi_{\mathcal{K}} \\[3pt]
    F_{\mathcal{K}}
  \end{bmatrix}
  \label{eq:cell-identity}
\end{equation}

\paragraph{Provenance hash.}
\[
  \Theta_{\mathcal{K}} =
  \mathcal{H}\!\bigl(\text{sources},\;
                      \text{recipe},\;
                      \text{version}\bigr)
\]

\paragraph{Pipeline / recipe identifier.}
\[
  \Pi_{\mathcal{K}} \in \Bset^{n}, \qquad n \in [64,\,256]
\]
(hash of ordered construction steps).

\paragraph{Source layer flag.}
\[
  F_{\mathcal{K}} \in \{Z,\,A,\,C\}
\]
matching the \texttt{XYZFormat} enumeration.

\medskip\noindent
This is the cell-level analogue of the particle provenance memory
$\Theta_i$, except it fingerprints \emph{structure lineage}, not a
single site.

% ·········································································
\subsubsection{Geometry \texorpdfstring{$\boldK^{\mathrm{geom}}$}{Kgeom}}
\label{sec:cell-geometry}

The cell geometry is defined by a lattice matrix and a boundary mode.

\paragraph{Lattice matrix.}
Let the lattice vectors be
$\mathbf{a},\,\mathbf{b},\,\mathbf{c} \in \R^3$.  Then:
%
\begin{equation}
  \boldL =
  \begin{bmatrix}
    \mathbf{a}^{\!\top} \\
    \mathbf{b}^{\!\top} \\
    \mathbf{c}^{\!\top}
  \end{bmatrix}
  \in \R^{3 \times 3}
  \label{eq:lattice-matrix}
\end{equation}
%
Cell volume: $\;V = \det(\boldL)$.

\paragraph{Boundary condition mode.}
\[
  \mathcal{B} \in \{\text{vac},\;\text{pbc},\;\text{slab}\}
\]

\noindent The geometry bundle is:
\begin{equation}
  \boldK^{\mathrm{geom}} =
  \bigl[\,
    \boldL,\;\;
    \mathcal{B},\;\;
    \mathbf{u}
  \,\bigr]
  \label{eq:cell-geometry}
\end{equation}
where $\mathbf{u}$ denotes units metadata (e.g.\ \AA).

\paragraph{Replication (supercell operator).}

For constructed cells:
\begin{equation}
  \mathbf{n} = (n_x,\,n_y,\,n_z) \in \N^3,
  \qquad
  \boldL' = \mathrm{diag}(\mathbf{n})\,\boldL
  \label{eq:supercell}
\end{equation}

% ·········································································
\subsubsection{Annotations \texorpdfstring{$\boldK^{\mathrm{comp}}$}{Kcomp}}
\label{sec:cell-comp}

Intentionally small and non-physical:
%
\begin{equation}
  \boldK^{\mathrm{comp}} =
  \begin{bmatrix}
    \Gamma_{\mathcal{K}} \\[3pt]
    \Xi_{\mathcal{K}} \\[3pt]
    \Upsilon_{\mathcal{K}}
  \end{bmatrix}
  \label{eq:cell-comp}
\end{equation}

\paragraph{Complexity class.}
$\;\Gamma_{\mathcal{K}} \in \{0,\,1,\,2,\,3\}$
\;(0: molecule, 1: small periodic, 2: bulk candidate,
   3: coupled / heterogeneous).

\paragraph{Kernel family selector.}
$\;\Xi_{\mathcal{K}} \in
  \{\text{pairwise},\;\text{screened},\;
   \text{EAM-like},\;\text{hybrid}\}$

\paragraph{Annotation confidence.}
$\;\Upsilon_{\mathcal{K}} = 1 - P(\text{misclassification})$

\bigskip\noindent
\textbf{Explicit exclusion:}
Runtime caches (bonds, neighbour lists, acceleration structures)
are not part of~$\mathcal{K}$; they live in a separate mutable
cache set:
%
\begin{equation}
  \mathcal{M}_{\mathcal{K}}(t) =
  \{\,\text{bonds},\;
     \text{neighbour lists},\;
     \text{BVH},\;
     \text{RDF bins},\;
     \ldots\,\}
  \label{eq:cell-cache}
\end{equation}

% ·········································································
\subsubsection{Contents}
\label{sec:cell-content}

A cell contains particles (or clusters) each with its own container:
\[
  \mathcal{K} \;\Rightarrow\;
  \{\mathcal{C}_i\}_{i=1}^{N}
\]
Optionally with a connectivity relation~$\mathcal{E}$ (bonds /
adjacency) treated as derived:
\begin{equation}
  \mathcal{E} = \mathrm{Infer}\!\bigl(
    \{\boldS_i\},\;
    \{\boldI_i\},\;
    \boldL,\;
    \mathcal{B}
  \bigr)
  \label{eq:connectivity}
\end{equation}

% ·········································································
\subsubsection{Implementation}
\label{sec:cell-comparison}

\begin{remark}[Alignment with \texttt{Crystal.hpp}]
The existing implementation already aligns with the container split:
%
\begin{center}
\renewcommand{\arraystretch}{1.3}
\begin{tabularx}{\linewidth}{>{\ttfamily}l c X}
  \toprule
  \textnormal{Code field} & $\longleftrightarrow$ & Formal layer \\
  \midrule
  xyz\_path, xyzA\_path, xyzC\_path
    & & $\boldK^{\mathrm{id}}$ (immutable references) \\
  lattice, replication
    & & $\boldK^{\mathrm{geom}}$ (geometry) \\
  ConstructionRecipe\{steps, hash\}
    & & $\Pi_{\mathcal{K}},\;\Theta_{\mathcal{K}}$ (recipe lineage) \\
  mutable inferred\_bonds, bonds\_computed
    & & $\mathcal{M}_{\mathcal{K}}(t)$ (caches, explicitly mutable) \\
  \bottomrule
\end{tabularx}
\end{center}
%
The notation imposes one discipline: caches never become ``truth.''
Humans enjoy turning caches into truth, then acting surprised when
it rots.
\end{remark}

If you keep your cell container this minimal, you get: deterministic
rebuild decisions, clean coarse-graining transitions, and provenance
you can actually defend in writing.


% ════════════════════════════════════════════════════════════════════════════
\subsection{Worlds}
\label{sec:world-container}
% ════════════════════════════════════════════════════════════════════════════

% ·········································································
\subsubsection{Definition}
\label{sec:world-def}

Define a world (a simulation ``instance'' that may contain many
cells, phases, or time segments) as:
%
\begin{equation}
  \mathcal{W} =
  \bigl(
    W^{\mathrm{id}},\;\;
    W^{\mathrm{top}},\;\;
    W^{\mathrm{ctrl}},\;\;
    W^{\mathrm{comp}}
  \bigr)
  \label{eq:world}
\end{equation}
%
\begin{itemize}
  \item $W^{\mathrm{id}}$: immutable provenance + dataset identity,
  \item $W^{\mathrm{top}}$: topology — cells, links, domains, boundaries,
  \item $W^{\mathrm{ctrl}}$: run controls — integrators, schedules, objectives,
  \item $W^{\mathrm{comp}}$: minimal compute annotation (not caches).
\end{itemize}

\noindent
Runtime caches are again excluded and live in
$\mathcal{M}_{\mathcal{W}}(t)$.

% ·········································································
\subsubsection{Identity \texorpdfstring{$W^{\mathrm{id}}$}{Wid}}
\label{sec:world-identity}

\begin{equation}
  W^{\mathrm{id}} =
  \bigl[\,
    \Theta_{\mathcal{W}},\;\;
    \Pi_{\mathcal{W}},\;\;
    D_{\mathcal{W}},\;\;
    \tau_{\mathcal{W}}
  \,\bigr]
  \label{eq:world-identity}
\end{equation}

\paragraph{World provenance hash.}
\begin{equation}
  \Theta_{\mathcal{W}} =
  \mathcal{H}\!\bigl(
    \{\Theta_{\mathcal{K}_m}\},\;
    \text{controls},\;
    \text{code version}
  \bigr)
  \label{eq:world-hash}
\end{equation}

\paragraph{Pipeline identifier.}
\[
  \Pi_{\mathcal{W}} \in \Bset^{n}, \qquad n \in [128,\,512]
\]

\paragraph{Dataset manifest.}
\begin{equation}
  D_{\mathcal{W}} =
  \bigl\{
    (\text{path}_j,\;\text{fmt}_j,\;\text{hash}_j)
  \bigr\}_{j=1}^{J}
  \label{eq:dataset-manifest}
\end{equation}

\paragraph{Timebase / epoch definition.}

Not ``\texttt{created\_utc}''—a real timebase:
\begin{equation}
  \tau_{\mathcal{W}} =
  \bigl(t_0,\;\Delta t,\;\mathcal{T}\bigr)
  \label{eq:timebase}
\end{equation}
where $\mathcal{T}$ is time domain type (continuous, bucketed,
event-driven).

% ·········································································
\subsubsection{Topology \texorpdfstring{$W^{\mathrm{top}}$}{Wtop}}
\label{sec:world-topology}

World contains a set of cells:
\[
  \{\mathcal{K}_m\}_{m=1}^{M}
\]
and a connectivity graph describing interactions / adjacency /
exchange between cells:
\begin{equation}
  G_{\mathcal{W}} = (V,\,E),
  \qquad V = \{1,\ldots,M\}
  \label{eq:world-graph}
\end{equation}

Each edge $e = (u,v) \in E$ carries a boundary operator:
\begin{equation}
  B_{uv} =
  \bigl(
    X_{uv},\;\;
    R_{uv},\;\;
    S_{uv}
  \bigr)
  \label{eq:boundary-operator}
\end{equation}
%
\begin{itemize}
  \item $X_{uv}$: exchange type (matter, charge, heat, constraints),
  \item $R_{uv}$: mapping operator (index map, ghost region,
        interpolation),
  \item $S_{uv}$: schedule (when exchange occurs).
\end{itemize}

\noindent Minimal world topology bundle:
\begin{equation}
  W^{\mathrm{top}} =
  \bigl[\,
    \{\mathcal{K}_m\},\;\;
    G_{\mathcal{W}}
  \,\bigr]
  \label{eq:world-top}
\end{equation}

\noindent
This is how you model multi-phase systems, slabs + vacuum cells,
reservoirs, defect domains, coarse-grained / atomistic coupling —
without duct-taping everything into one mega-cell.

% ·········································································
\subsubsection{Controls \texorpdfstring{$W^{\mathrm{ctrl}}$}{Wctrl}}
\label{sec:world-ctrl}

Controls are not ``global variables''; they are a formal program.
%
\begin{equation}
  W^{\mathrm{ctrl}} =
  \bigl[\,
    \mathcal{I},\;\;
    \mathcal{O},\;\;
    \mathcal{S},\;\;
    \mathcal{A}
  \,\bigr]
  \label{eq:world-ctrl}
\end{equation}

\paragraph{Integrator family.}
$\;\mathcal{I} \in \{\text{MD},\;\text{MC},\;\text{hybrid},\;\text{relax}\}$

\paragraph{Objectives (scoring / constraints).}
\begin{equation}
  \mathcal{O} = \{O_k\}_{k=1}^{K},
  \qquad O_k : \mathcal{W} \to \R
  \label{eq:objectives}
\end{equation}

\paragraph{Schedule (what runs when).}
\begin{equation}
  \mathcal{S} = \{(t_j,\;\mathrm{op}_j)\}_{j=1}^{J}
  \label{eq:schedule}
\end{equation}

\paragraph{Audit policy (self-checks).}
\begin{equation}
  \mathcal{A} = \{A_\ell\}_{\ell=1}^{L},
  \qquad A_\ell : \mathcal{W} \to \{0,\,1\}
  \label{eq:audit-policy}
\end{equation}

\noindent
This makes the always-on self-auditing system first-class, not an
afterthought.

% ·········································································
\subsubsection{Annotations \texorpdfstring{$W^{\mathrm{comp}}$}{Wcomp}}
\label{sec:world-comp}

\begin{equation}
  W^{\mathrm{comp}} =
  \begin{bmatrix}
    \Gamma_{\mathcal{W}} \\[3pt]
    \Xi_{\mathcal{W}} \\[3pt]
    \Upsilon_{\mathcal{W}}
  \end{bmatrix}
  \label{eq:world-comp}
\end{equation}

Complexity class, kernel family selector, annotation confidence.
Same pattern as particle and cell.
Consistency is boring, which is why it works.

% ·········································································
\subsubsection{Caches}
\label{sec:world-cache}

\begin{equation}
  \mathcal{M}_{\mathcal{W}}(t) =
  \{\,\text{neighbour lists},\;
     \text{domain decompositions},\;
     \text{GPU buffers},\;
     \text{BVHs},\;
     \text{RDF histograms},\;
     \text{FFT plans},\;
     \ldots\,\}
  \label{eq:world-cache}
\end{equation}

These are derivable artifacts.  If you serialise them, it is only as
optional acceleration, never as truth.


% ════════════════════════════════════════════════════════════════════════════
%  0.3.7 — Nano–Meso Boundary Extensions
% ════════════════════════════════════════════════════════════════════════════

\bigskip
\subsubsection*{0.3.7\quad Nano–Meso Boundary Extensions}
\addcontentsline{toc}{subsubsection}{0.3.7 Nano–Meso Boundary Extensions}
\label{sec:nano-meso}

% ·········································································
\paragraph{0.3.7.1\quad Surface–Interior Decomposition.}
\label{sec:surface-interior}
\mbox{}\\[0.4em]

For any cell or domain $\mathcal{K}$ containing $N$ sites, define a
partition into surface and interior subsets:
\begin{equation}
  \mathcal{K} = \mathcal{K}^{\partial}
  \cup \mathcal{K}^{\circ},
  \qquad
  \mathcal{K}^{\partial}
  \cap \mathcal{K}^{\circ} = \varnothing
  \label{eq:surface-split}
\end{equation}

A minimal deterministic definition uses a local coordination deficit:
\begin{equation}
  \delta_i = z_{\text{bulk}}(Z_i,\,\Sigma_i) - z_i
  \label{eq:coord-deficit}
\end{equation}
where $z_i$ is measured coordination, and $z_{\text{bulk}}$ is the
expected interior coordination for species class
$(Z_i,\,\Sigma_i)$.  Then:
\[
  i \in \mathcal{K}^{\partial}
  \;\Longleftrightarrow\;
  \delta_i \geq \delta_{\min},
  \qquad
  \text{else } i \in \mathcal{K}^{\circ}
\]

This split matters because the surface carries:
\begin{itemize}
  \item dominant chemical reactivity,
  \item defect nucleation and healing,
  \item adsorption and morphology selection,
\end{itemize}
while the interior carries:
\begin{itemize}
  \item bulk stiffness,
  \item long-time stability regimes,
  \item lattice ordering and transport baselines.
\end{itemize}

\noindent
Coarse graining that erases this split produces beads with
``average'' behaviour that matches nothing.

% ·········································································
\paragraph{0.3.7.2\quad Formation-Condition Determinism as a
  First-Class Map.}
\label{sec:formation-determinism}
\mbox{}\\[0.4em]

At the nano–meso boundary, emergent structure is highly sensitive to
formation conditions.  The design goal is not to ``fit'' outcomes,
but to model a deterministic projection from formation controls to
mesoscopic identity.

Define a formation condition vector:
\begin{equation}
  \mathcal{E} =
  \bigl[\,
    T(t),\;\;
    c(t),\;\;
    p(t),\;\;
    P,\;\;
    R
  \,\bigr]
  \label{eq:formation-conditions}
\end{equation}
where:
\begin{itemize}
  \item $T(t)$ is the thermal history (dominant driver),
  \item $c(t)$ is composition / concentration history,
  \item $p(t)$ is pressure history (or constraint proxy),
  \item $P$ denotes pathway class (reaction / assembly route),
  \item $R$ denotes environmental regime (solvent, vacuum, ionic
        melt, etc.).
\end{itemize}

The core postulate for the boundary scale is:
\begin{equation}
  \boxed{\;
    M(t^{*}) = \mathcal{F}\!\bigl(\mathcal{E}\bigr)
  \;}
  \label{eq:formation-map}
\end{equation}
where $M(t^{*})$ is a mesoscopic descriptor at the end of formation
(or at a chosen observation epoch~$t^{*}$).

This is framed positively because it is exactly what makes the
nano–meso boundary scientifically valuable: formation is not noise,
it is structure selection.

% ·········································································
\paragraph{0.3.7.3\quad Lower Meso Descriptor (Precise Boundary
  Definition).}
\label{sec:lower-meso}
\mbox{}\\[0.4em]

Define ``lower meso'' (nano–meso boundary) as the regime where:
\begin{itemize}
  \item individual atoms are no longer the preferred degrees of
        freedom,
  \item yet the structure is not fully bulk-continuum.
\end{itemize}

Operationally, represent the system as domains $D_k$ built from
atoms:
\[
  D_k \subset \{1,\ldots,N\}, \qquad k = 1,\ldots,K
\]

Each domain carries a minimal descriptor vector:
\begin{equation}
  M_k =
  \bigl[\,
    \phi_k,\;\;
    \psi_k,\;\;
    \chi_k,\;\;
    \omega_k,\;\;
    E_k
  \,\bigr]
  \label{eq:meso-descriptor}
\end{equation}
where:

\paragraph{Surface fraction.}
\begin{equation}
  \phi_k = \frac{|D_k^{\partial}|}{|D_k|}
  \label{eq:surface-fraction}
\end{equation}

\paragraph{Ordering / crystallinity proxy.}
Any stable scalar order parameter:
\begin{equation}
  \psi_k = \langle \Psi(\mathbf{r}_i) \rangle_{i \in D_k}
  \label{eq:ordering}
\end{equation}

\paragraph{Defect density.}
\begin{equation}
  \chi_k = \frac{1}{|D_k|}
  \sum_{i \in D_k}
  \mathbf{1}[\delta_i \geq \delta_{\min}]
  \label{eq:defect-density}
\end{equation}

\paragraph{Composition / identity mean.}
\begin{equation}
  \omega_k = \sum_{i \in D_k} w_i\, \boldI_i
  \label{eq:comp-mean}
\end{equation}

\paragraph{Identity entropy (heterogeneity retained).}
\begin{equation}
  E_k = \Var(\boldI_i)_{i \in D_k}
  \label{eq:identity-entropy}
\end{equation}

\medskip\noindent
This is the boundary-scale sweet spot: you can now distinguish
\begin{itemize}
  \item uniform crystal domains vs.\ mixed disordered aggregates,
  \item smooth surfaces vs.\ high-defect catalytic skins,
  \item pathway-selected polymorphs vs.\ temperature-quenched glassy
        states.
\end{itemize}

% ·········································································
\paragraph{0.3.7.4\quad Temperature as the Dominant Deterministic
  Driver.}
\label{sec:temperature-driver}
\mbox{}\\[0.4em]

Treat temperature not as a scalar input but as a history operator:
\begin{equation}
  T(t) \;\Rightarrow\;
  \beta(t) = \frac{1}{k_B\, T(t)}
  \label{eq:inverse-temp}
\end{equation}

Formation and relaxation become biased flows on the state manifold:
\begin{equation}
  \frac{d\boldS}{dt} =
  -\nabla U(\boldS;\,\boldI) + \eta(t)
  \label{eq:biased-flow}
\end{equation}
but in deterministic coarse graining, the randomness term $\eta(t)$
is replaced by a reproducible stencil driven by the provenance hash:
\begin{equation}
  \eta(t) \equiv
  \mathcal{G}\!\bigl(\mathcal{H}(\boldI,\,\boldS,\,t)\bigr)
  \label{eq:deterministic-noise}
\end{equation}

So the mapping $\mathcal{F}(\mathcal{E})$ remains stable across
reruns.

This is precisely why temperature is ``the main one'' at this
boundary: thermal history controls which basins are accessible, which
defects persist, and whether the surface freezes into metastable
motifs.

% ·········································································
\paragraph{0.3.7.5\quad Emergence Is a Feature, Not a Problem.}
\label{sec:emergence}
\mbox{}\\[0.4em]

The hard part is not that formation conditions influence outcome.
The hard part is doing it without losing determinism and without
lying about coarse-grained meaning.

This framework treats emergence as a structured output:
\[
  \{M_k\}_{k=1}^{K} \;\text{is the emergence layer}
\]
and preserves the full trace:
\begin{equation}
  \mathcal{E}
  \;\longrightarrow\;
  M
  \;\longrightarrow\;
  \text{CG beads / fields}
  \label{eq:emergence-trace}
\end{equation}

So rather than coarse graining being a destructive step, it becomes
the formal bridge between formation physics and mesoscopic realism.

% ════════════════════════════════════════════════════════════════════════════
\vfill
\begin{center}
  \rule{0.4\textwidth}{0.4pt}\\[0.5em]
  \textit{Formation Engine Methodology v0.1 —
          End of §0 (Pre-Section Theory Layer)}
\end{center}

\end{document}
