\documentclass[11pt]{article}
\usepackage{amsmath,amssymb,amsthm}
\usepackage{geometry}
\usepackage{booktabs}
\usepackage{graphicx}
\usepackage{hyperref}
\usepackage{enumerate}
\usepackage{listings}
\usepackage{xcolor}

\geometry{letterpaper, margin=1in}

\lstset{
  basicstyle=\ttfamily\small,
  keywordstyle=\color{blue},
  commentstyle=\color{gray},
  stringstyle=\color{red},
  breaklines=true,
  columns=fullflexible,
  keepspaces=true
}

\title{\textbf{Formation Engine Methodology} \\
\Large Section 6 — Formation Physics and Emergent Structure}
\author{Formation Engine Development Team}
\date{Version 0.1 — As Implemented in Codebase}

\begin{document}
\maketitle

\section{Introduction: What is a Formation?}

A \textbf{formation} is a local minimum of the potential energy surface reached through a physically motivated search procedure. This section documents how the formation engine:

\begin{enumerate}
\item Initializes molecular geometries (VSEPR prediction)
\item Evolves systems through dynamics (MD exploration)
\item Identifies stable configurations (FIRE quenching)
\item Classifies and scores results (pattern recognition)
\end{enumerate}

Everything documented here \textbf{exists in the codebase} with file locations and line numbers.

\section{VSEPR Geometry Prediction}

\textbf{Implementation:} \texttt{src/pot/energy\_vsepr.hpp}, class \texttt{VSEPREnergy}

Before running molecular dynamics, the engine predicts an initial geometry using Valence Shell Electron Pair Repulsion (VSEPR) theory.

\subsection{The VSEPR Model}

VSEPR theory states that electron domains (bonding pairs and lone pairs) around a central atom arrange themselves to minimize electrostatic repulsion. This is modeled as a classical potential energy:

\begin{equation}
U_{\text{VSEPR}} = k_{\text{VSEPR}} \sum_{a<b} \frac{w_{ab}}{[\varepsilon + (1 - \cos\theta_{ab})]^p}
\end{equation}

where:
\begin{itemize}
\item $\theta_{ab}$ is the angle between domains $a$ and $b$
\item $w_{ab}$ is a weight depending on domain types
\item $p$ is a stiffness exponent (default: 1.5)
\item $\varepsilon$ is a regularization parameter to prevent singularity
\end{itemize}

\subsection{Domain Representation}

From \texttt{energy\_vsepr.hpp}, lines 24--29:

\begin{lstlisting}[language=C++]
struct Domain {
    uint32_t central_atom;  // Central atom index
    bool is_lone_pair;      // True for LP, false for BP
    uint32_t partner_atom;  // For BP: bonded atom
                            // For LP: LP direction index
};
\end{lstlisting}

\paragraph{Bonding pairs (BP)} are represented by the vector from the central atom to its bonded neighbor.

\paragraph{Lone pairs (LP)} are represented as virtual sites at distance $r_0 = 0.5$\,\AA\ from the central atom:
\begin{equation}
\mathbf{r}_{\text{LP}} = \mathbf{r}_{\text{central}} + r_0 \cdot \mathbf{u}
\end{equation}

where $\mathbf{u}$ is a unit vector optimized during geometry minimization.

\subsection{Domain Weights}

The repulsion strength depends on domain types (lines 13--15):

\begin{lstlisting}[language=C++]
double w_LP_LP = 2.0;   // Lone pair - lone pair weight
double w_LP_BP = 1.5;   // Lone pair - bond pair weight  
double w_BP_BP = 1.0;   // Bond pair - bond pair weight
\end{lstlisting}

This encodes the empirical observation that:
\begin{itemize}
\item LP--LP repulsion is strongest (2.0$\times$)
\item LP--BP repulsion is intermediate (1.5$\times$)
\item BP--BP repulsion is weakest (1.0$\times$)
\end{itemize}

\subsection{Geometry Optimization}

The VSEPR energy is minimized using a gradient-based optimizer (lines 39--79). The gradient with respect to domain angle $\theta$ is:

\begin{equation}
\frac{\partial U}{\partial \cos\theta} = \frac{k_{\text{VSEPR}} \cdot w \cdot p}{[\varepsilon + (1 - \cos\theta)]^{p+1}}
\end{equation}

For bonding pairs, this gradient propagates to atomic positions. For lone pairs, it propagates to the unit vector $\mathbf{u}$ representing the lone pair direction.

\subsection{Typical VSEPR Geometries}

\begin{center}
\begin{tabular}{lcccl}
\toprule
Domains & BP & LP & Geometry & Example \\
\midrule
2 & 2 & 0 & Linear & CO$_2$ \\
3 & 3 & 0 & Trigonal planar & BF$_3$ \\
3 & 2 & 1 & Bent & SO$_2$ \\
4 & 4 & 0 & Tetrahedral & CH$_4$ \\
4 & 3 & 1 & Trigonal pyramidal & NH$_3$ \\
4 & 2 & 2 & Bent & H$_2$O \\
5 & 5 & 0 & Trigonal bipyramidal & PCl$_5$ \\
6 & 6 & 0 & Octahedral & SF$_6$ \\
\bottomrule
\end{tabular}
\end{center}

\subsection{Limitations}

\paragraph{No explicit electron counting} The engine does not compute the number of lone pairs from Lewis structures or valence electron counts. The user must specify the number of lone pairs manually or infer them from bond count and valence.

\paragraph{Single central atom} The current implementation optimizes domains around one central atom at a time. Multi-center molecules (e.g., ethane C$_2$H$_6$) require sequential application to each carbon.

\paragraph{Coordination number $> 6$} VSEPR theory becomes less reliable for coordination numbers above 6 (rare main-group molecules).

\section{Formation Detection via MD + Quench}

\textbf{The formation pipeline:} Documented in Section~5, implemented across multiple integrators

\subsection{The Two-Stage Process}

\begin{enumerate}
\item \textbf{Exploration (MD):} Velocity Verlet or Langevin dynamics at temperature $T$ for $n_{\text{steps}}$ steps (typical: 1000--10,000 steps at 1\,fs timestep)

\item \textbf{Quenching (FIRE):} Extract snapshots at intervals (every 100 steps), minimize each with FIRE to convergence
\end{enumerate}

The MD stage provides thermal energy to cross barriers. The FIRE stage projects away thermal fluctuations, revealing the underlying local minima.

\subsection{Why This Works}

\paragraph{Barrier crossing} At finite temperature, the system has kinetic energy $\sim \frac{3}{2}Nk_BT$. For $N = 100$ atoms at $T = 300$\,K, this is $\sim 90$\,kcal/mol. This is sufficient to cross small barriers (1--5\,kcal/mol) but not large barriers (> 20\,kcal/mol).

\paragraph{Basin identification} FIRE minimization is deterministic: given the same starting configuration, it always converges to the same minimum (modulo numerical precision). Different MD snapshots sample different basins, and FIRE identifies which basin each snapshot belongs to.

\paragraph{Metastable states} The method finds \textit{all} local minima accessible at temperature $T$, not just the global minimum. This is physically correct: many molecular systems have multiple metastable conformers (e.g., butane has trans, gauche+, gauche$-$).

\subsection{Formation Convergence Criteria}

From Section~5.3, a formation is considered \textbf{converged} when FIRE terminates with:

\begin{align}
F_{\text{rms}} &< 10^{-6}\,\text{kcal/(mol·\AA)} \quad \text{OR} \\
\frac{|\Delta U|}{N} &< 10^{-10}\,\text{kcal/mol per atom}
\end{align}

A formation is considered \textbf{stable} if it remains at the same minimum after small perturbations ($ \pm 0.1$\,\AA) followed by re-minimization.

\section{Molecular Classification System}

\textbf{Implementation:} \texttt{tools/classification.py}, class \texttt{MolecularClassifier}

Every formation is classified into one or more chemical categories based on elemental composition. This classification drives the scoring bonus $w_C$ (Section~7).

\subsection{Classification Rules}

From \texttt{classification.py}, lines 15--24:

\begin{lstlisting}[language=Python]
"""
Classifications (priority order):
1. Organometallic (+1.0 bonus) - HIGHEST IMPORTANCE
2. Perfluorinated (+0.8) - 6+ fluorines
3. Semiconductor (+0.7)
4. Metallic superalloy (+0.6)
5. Organic super acid (+0.5)
6. Organic super base (+0.5)
7. Explosive (+0.3) - flagged for caution
"""
\end{lstlisting}

Each classification has:
\begin{itemize}
\item \textbf{Label}: Human-readable name
\item \textbf{Confidence}: 0.0--1.0 (currently binary: 0 or 1)
\item \textbf{Bonus}: Scoring multiplier (Section~6.5)
\item \textbf{Reason}: Why the classification was applied
\end{itemize}

\subsection{Organometallic Detection}

\textbf{Implementation:} \texttt{classification.py}, lines 109--131

\paragraph{Rule:} Contains both carbon \textbf{and} at least one metal element.

\paragraph{Examples:}
\begin{itemize}
\item Grignard reagents: CH$_3$MgBr
\item Ferrocene: C$_{10}$H$_{10}$Fe
\item Organolithium: C$_4$H$_9$Li
\end{itemize}

Code (lines 118--131):

\begin{lstlisting}[language=Python]
has_carbon = 'C' in elements
has_metal = bool(elements & self.metals)

if has_carbon and has_metal:
    metal_list = sorted(elements & self.metals)
    return Classification(
        label="organometallic",
        confidence=1.0,
        bonus=1.0,
        reason=f"Contains C + metals ({', '.join(metal_list)})"
    )
\end{lstlisting}

The bonus of $+1.0$ is the \textbf{highest} in the scoring system, reflecting the scientific importance of organometallic chemistry.

\subsection{Perfluorinated Detection}

\textbf{Implementation:} \texttt{classification.py}, lines 133--145

\paragraph{Rule:} Contains 6 or more fluorine atoms.

\paragraph{Examples:}
\begin{itemize}
\item Teflon repeat unit: C$_2$F$_4$
\item Perfluorooctanoic acid: C$_8$HF$_{15}$O$_2$
\end{itemize}

Code:

\begin{lstlisting}[language=Python]
n_F = counts.get('F', 0)
if n_F >= 6:
    return Classification(
        label="perfluorinated",
        confidence=1.0,
        bonus=0.8,
        reason=f"Contains {n_F} fluorine atoms"
    )
\end{lstlisting}

Perfluorinated compounds are chemically inert, thermally stable, and used in specialty applications (lubricants, coatings, electronics).

\subsection{Semiconductor Detection}

\paragraph{Rule:} Contains at least one semiconductor element (Si, Ge, Ga, As, In, Sb, Te) \textbf{and} at least 10\% composition by atom count.

\paragraph{Examples:}
\begin{itemize}
\item Silicon carbide: SiC
\item Gallium arsenide: GaAs
\item Indium phosphide: InP
\end{itemize}

Bonus: $+0.7$

\subsection{Metallic Superalloy Detection}

\paragraph{Rule:} Contains at least 3 superalloy metals from the set \{Ni, Co, Cr, W, Mo, Re, Ta, Nb, Ti, Al\}.

\paragraph{Examples:}
\begin{itemize}
\item Inconel (Ni-Cr-Fe alloys)
\item Rene alloys (Ni-Co-Cr-W-Mo)
\end{itemize}

Bonus: $+0.6$

\subsection{Super Acid/Base Detection}

\paragraph{Super acid rule:} Contains organic elements + acid-forming elements (S, P, N, Cl, Br, I) with high acid element fraction.

\paragraph{Super base rule:} Contains organic elements + nitrogen with high nitrogen fraction.

\paragraph{Examples:}
\begin{itemize}
\item Triflic acid: CF$_3$SO$_3$H
\item Guanidine: (NH$_2$)$_2$C=NH
\end{itemize}

Bonus: $+0.5$ each

\subsection{Explosive Detection}

\paragraph{Rule:} Contains all three elements \{N, O, Cl\} with high N+O fraction (> 50\%).

\paragraph{Examples:}
\begin{itemize}
\item TNT: C$_7$H$_5$N$_3$O$_6$
\item Nitroglycerin: C$_3$H$_5$N$_3$O$_9$
\end{itemize}

Bonus: $+0.3$ (lower because explosives are hazardous, not scientifically interesting per se)

\textbf{Note:} This is a \textit{flagging} mechanism, not a physical explosion predictor. The simulation does not model detonation.

\section{Formation Scoring System}

\textbf{Implementation:} \texttt{tools/scoring.py}, class \texttt{MolecularScorer}

Every formation is assigned a priority score $S \in [0, 100]$ that determines its ranking in the discovery pipeline.

\subsection{The Master Scoring Formula}

From \texttt{scoring.py}, lines 30--33:

\begin{lstlisting}[language=Python]
"""
Score = V / (1 + λC)
where V = w_N · w_Q · w_M · w_D · w_S · w_C
"""
\end{lstlisting}

The score balances \textbf{scientific value} $V$ against \textbf{computational cost} $C$. The multiplicative structure means \textit{any single disqualifying factor} (charge imbalance, explosion, low diversity) drags the entire score toward zero.

\subsection{Factor 1: Size Preference ($w_N$)}

\textbf{Implementation:} Lines 67--76

A dual-Gaussian curve with peaks at $N = 8$ (small molecules) and $N = 56$ (exploratory bonding stage):

\begin{equation}
w_N = a \cdot \exp\!\left(-\frac{(N - \mu_1)^2}{2\sigma_1^2}\right) + b \cdot \exp\!\left(-\frac{(N - \mu_2)^2}{2\sigma_2^2}\right)
\end{equation}

Default parameters:
\begin{itemize}
\item $\mu_1 = 8$, $\sigma_1 = 3$, $a = 1.0$
\item $\mu_2 = 56$, $\sigma_2 = 12$, $b = 0.8$
\end{itemize}

\paragraph{Rationale:} Small molecules ($N \sim 8$) are chemically well-defined (water, ammonia, methane). Medium-sized clusters ($N \sim 56$) exhibit rich bonding patterns not captured by simple molecules or bulk crystals.

\subsection{Factor 2: Charge Neutrality ($w_Q$)}

\textbf{Implementation:} Lines 78--87

\begin{equation}
w_Q = \exp(-k \cdot |\sum_i q_i|)
\end{equation}

Default: $k = 2.0$

\paragraph{Rationale:} Bulk and crystalline systems must be electrically neutral. A net charge of $+1\,e$ reduces the score by $\sim 86\%$ ($e^{-2} \approx 0.135$). Molecular ions (CH$_3^+$, OH$^-$) are allowed but penalized.

\subsection{Factor 3: Metal Richness ($w_M$)}

\textbf{Implementation:} Lines 89--114

\begin{equation}
w_M = 1 + \alpha \cdot \frac{N_{\text{metal}}}{N}
\end{equation}

Default: $\alpha = 0.3$

\paragraph{Rationale:} Metal-containing systems (alloys, intermetallics, organometallics) are scientifically interesting. A system with 50\% metals gets a 15\% bonus ($w_M = 1.15$).

\subsection{Factor 4: Element Diversity ($w_D$)}

\textbf{Implementation:} Lines 116--129

\begin{equation}
w_D = 1 + \beta \cdot \frac{n_{\text{unique}}}{\ln(1 + N)}
\end{equation}

Default: $\beta = 0.5$

\paragraph{Rationale:} Systems with diverse elements (ternaries, quaternaries) are more interesting than trivial duplicates (C$_{100}$). The logarithmic normalization prevents large systems from dominating solely due to size.

\subsection{Factor 5: Stability Gate ($w_S$)}

\textbf{Implementation:} Lines 167--182

A categorical factor based on simulation health:

\begin{equation}
w_S = \begin{cases}
1.0 & \text{if bounded and converged} \\
0.3 & \text{if bounded but not converged} \\
0.05 & \text{if exploded}
\end{cases}
\end{equation}

\paragraph{Rationale:} Unstable systems (explosions, NaN forces) are not useful formations. Partially converged systems may still provide structural insight, hence $w_S = 0.3$ instead of 0.

\subsection{Factor 6: Classification Bonus ($w_C$)}

\textbf{Implementation:} Lines 131--148

\begin{equation}
w_C = 1 + \sum_{\text{classifications}} \text{bonus}
\end{equation}

Multiple classifications stack additively:
\begin{itemize}
\item Organometallic C$_5$H$_5$Fe: $w_C = 1.0 + 1.0 = 2.0$ (2$\times$ multiplier)
\item Perfluorinated semiconductor SiF$_8$: $w_C = 1.0 + 0.8 + 0.7 = 2.5$
\end{itemize}

\subsection{Computational Cost ($C$)}

\textbf{Implementation:} Lines 184--200

\begin{equation}
C = \begin{cases}
N^2 & \text{if long-range interactions (Coulomb, PME)} \\
N^{1.5} & \text{otherwise}
\end{cases}
\end{equation}

Normalized to $[0, 1]$ range. Default $\lambda = 0.01$ (cost penalty is mild).

\paragraph{Rationale:} Large systems take longer to simulate. The $N^2$ scaling reflects the all-pairs force evaluation cost. The cost penalty prevents the pipeline from wasting time on $N = 10{,}000$ trivial systems.

\subsection{Final Score}

\begin{equation}
S = \frac{w_N \cdot w_Q \cdot w_M \cdot w_D \cdot w_S \cdot w_C}{1 + \lambda C} \times 50
\end{equation}

The factor of 50 scales the score to $[0, 100]$ range (approximately).

\subsection{Numerical Example}

Consider ferrocene (C$_{10}$H$_{10}$Fe) at 300\,K, converged:

\begin{align}
N &= 21 \\
w_N &= 0.95 \quad \text{(close to peak at } N = 8\text{)} \\
w_Q &= 1.0 \quad \text{(neutral)} \\
w_M &= 1 + 0.3 \times (1/21) = 1.014 \quad \text{(1 metal atom)} \\
w_D &= 1 + 0.5 \times (3 / \ln 22) = 1.485 \quad \text{(3 elements: C, H, Fe)} \\
w_S &= 1.0 \quad \text{(converged)} \\
w_C &= 2.0 \quad \text{(organometallic bonus)} \\
C &= 21^{1.5} = 96.23 \; \Rightarrow \; \frac{1}{1 + 0.01 \times 96.23} = 0.509 \\
S &= 0.95 \times 1.0 \times 1.014 \times 1.485 \times 1.0 \times 2.0 \times 0.509 \times 50 \\
&\approx 72.3
\end{align}

This is a \textbf{high-priority formation} (score > 70).

\subsection{Score Breakdown Transparency}

Every formation stores its complete score breakdown:

\begin{lstlisting}[language=Python]
@dataclass
class ScoreBreakdown:
    wN: float      # Size preference
    wQ: float      # Charge neutrality
    wM: float      # Metal richness
    wD: float      # Element diversity
    wS: float      # Stability gate
    wC: float      # Classification bonus
    cost: float    # Computational cost
    value: float   # Scientific value (V)
    priority: float  # Final priority score (S)
    classifications: List[str]  # Applied labels
\end{lstlisting}

This allows \textbf{post-hoc analysis}: "Why did formation X score higher than Y?" The answer is explicit, not a black-box neural network.

\section{Basin Mapping and Conformer Ensembles}

\textbf{Status:} Conceptual framework defined, not fully implemented

\subsection{The Basin Concept}

A \textbf{basin of attraction} is the set of all initial configurations from which FIRE minimization converges to the same local minimum. The basin landscape determines which formations are accessible and how likely they are to be observed.

\paragraph{Funnel-like landscape} One dominant basin. All starting configurations converge to the same minimum. The formation is robust to perturbations.

\paragraph{Multi-basin landscape} Multiple competing minima. Different initial conditions lead to different formations. The system has multiple metastable conformers.

\subsection{Basin Probing Protocol}

The adaptive sampling mode maps the basin landscape:

\begin{enumerate}
\item Start from a reference structure (e.g., VSEPR-predicted geometry)
\item Perturb all positions by Gaussian noise: $\mathbf{r}_i \gets \mathbf{r}_i + \mathcal{N}(0, \sigma^2)$ with $\sigma = 0.2$\,\AA
\item FIRE-minimize the perturbed structure
\item Record the final energy $E_{\text{final}}$
\item Repeat $n_{\text{samples}}$ times (default: 100)
\item Analyze the energy distribution
\end{enumerate}

\subsection{Convergence Detection}

The stationarity gate (Section~7.2) determines when the energy distribution has converged:

\begin{equation}
|E_{\text{new}} - \bar{E}| < 3\sigma + \varepsilon_{\text{mean}}
\end{equation}

If all samples converge to the same energy (within tolerance), the formation is in a single basin. If multiple distinct energies emerge, there are competing basins.

\subsection{Conformer Ensemble Generation}

For flexible molecules, larger perturbations ($\sigma = 0.3$\,\AA) sample different rotameric states. Conformers are clustered by RMSD after Kabsch alignment (Section~7.3):

\begin{enumerate}
\item Align all structures to a reference (remove translation and rotation)
\item Compute pairwise RMSD:
\begin{equation}
\text{RMSD}_{ij} = \sqrt{\frac{1}{N}\sum_{k=1}^{N} |\mathbf{r}_k^{(i)} - \mathbf{r}_k^{(j)}|^2}
\end{equation}
\item Cluster by RMSD threshold (default: 0.5\,\AA)
\item Structures with RMSD $< 0.5$\,\AA are the same conformer
\end{enumerate}

\subsection{Boltzmann Populations}

The thermodynamic weight of conformer $i$ at temperature $T$ is:

\begin{equation}
p_i = \frac{e^{-E_i / k_B T}}{\sum_j e^{-E_j / k_B T}}
\end{equation}

This connects the static formation results back to thermodynamics: the probability of observing each conformer at equilibrium.

\paragraph{Example:} Butane at 300\,K has three conformers:
\begin{itemize}
\item Trans: $E = 0.0$\,kcal/mol, $p = 0.64$
\item Gauche+: $E = 0.9$\,kcal/mol, $p = 0.18$
\item Gauche$-$: $E = 0.9$\,kcal/mol, $p = 0.18$
\end{itemize}

The trans conformer is dominant (64\% population) but the gauche conformers are still observable (18\% each).

\subsection{Current Implementation Status}

\paragraph{Implemented:}
\begin{itemize}
\item FIRE minimization (Section~5.3)
\item Stationarity gate for convergence detection (Section~7.2)
\item Kabsch alignment for structural comparison (Section~7.3)
\item Energy-based scoring (Section~6.5)
\end{itemize}

\paragraph{Not yet implemented:}
\begin{itemize}
\item Automated basin probing workflow
\item Conformer clustering by RMSD
\item Boltzmann population calculation
\item Multi-basin visualization
\end{itemize}

These are future extensions. The underlying tools exist; the integration is pending.

\section{Bond Inference from Geometry}

\textbf{Implementation:} Documented in Section~2.3 (State Persistence), implemented in xyzA format parser

Bonds are not user-specified. They are \textbf{inferred} from interatomic distances using covalent radii.

\subsection{The Inference Rule}

From Section~2.3:

\begin{equation}
\text{bonded}(i, j) \;\Longleftrightarrow\; |\mathbf{r}_i - \mathbf{r}_j| \;<\; f \cdot \bigl(r_{\text{cov},i}^{(k)} + r_{\text{cov},j}^{(k)}\bigr)
\end{equation}

where:
\begin{itemize}
\item $f \approx 1.2$ is a tolerance factor
\item $k \in \{1, 2, 3\}$ selects the bond order (single, double, triple)
\end{itemize}

The bond order is chosen to minimize $|r_{ij} - (r_{\text{cov},i}^{(k)} + r_{\text{cov},j}^{(k)})|$.

\subsection{Covalent Radii Database}

From the periodic table (Section~1.3), each element has three covalent radii:

\begin{center}
\small
\begin{tabular}{lccc}
\toprule
Element & $r_{\text{cov}}^{(1)}$ (\AA) & $r_{\text{cov}}^{(2)}$ (\AA) & $r_{\text{cov}}^{(3)}$ (\AA) \\
\midrule
H & 0.31 & — & — \\
C & 0.76 & 0.67 & 0.60 \\
N & 0.71 & 0.62 & 0.56 \\
O & 0.66 & 0.57 & — \\
F & 0.57 & — & — \\
Cl & 1.02 & 0.95 & — \\
\bottomrule
\end{tabular}
\end{center}

\subsection{Numerical Example: Ethene (C$_2$H$_4$)}

Consider two carbon atoms at distance $r_{\text{CC}} = 1.34$\,\AA.

\paragraph{Single bond check:}
\begin{equation}
1.2 \times (0.76 + 0.76) = 1.824\,\text{\AA} \quad \text{(too large)}
\end{equation}

\paragraph{Double bond check:}
\begin{equation}
1.2 \times (0.67 + 0.67) = 1.608\,\text{\AA} \quad \text{(close)}
\end{equation}

The distance $1.34$\,\AA $< 1.608$\,\AA, so a C=C double bond is inferred.

\subsection{Limitations}

\paragraph{Strained systems} Cyclopropane (C$_3$H$_6$) has C--C bonds at $\sim 1.51$\,\AA, which are slightly longer than normal single bonds due to ring strain. The $f = 1.2$ tolerance accommodates this.

\paragraph{Metallic clusters} Metal--metal bonds in clusters (e.g., Fe$_3$) have variable lengths depending on coordination. The inference rule may misassign bond orders.

\paragraph{No Lewis structure} The engine does not compute formal charges, resonance structures, or aromaticity. Bonds are purely geometric.

\section{Formation as Emergence: The Water Example}

\textbf{The test case:} H$_2$O

The formation engine does not store the fact that water is bent at 104.5$^\circ$ or that the O--H bond length is 0.96\,\AA. It knows only:
\begin{itemize}
\item Oxygen: $Z = 8$, $m = 15.999$\,amu, $r_{\text{cov}}^{(1)} = 0.66$\,\AA, $\chi = 3.44$
\item Hydrogen: $Z = 1$, $m = 1.008$\,amu, $r_{\text{cov}}^{(1)} = 0.31$\,\AA, $\chi = 2.20$
\item Lennard-Jones parameters (UFF)
\item Bonded force field (harmonic bonds, angles)
\end{itemize}

If the force field and VSEPR prediction are correct, the 104.5$^\circ$ angle and 0.96\,\AA bond length must \textbf{emerge} from the simulation.

\subsection{The Formation Protocol for Water}

\begin{enumerate}
\item \textbf{Initialize:} Place 1 oxygen + 2 hydrogens randomly in a $5 \times 5 \times 5$\,\AA\ box

\item \textbf{VSEPR prediction:} Infer 2 bonding pairs + 2 lone pairs $\to$ tetrahedral electron geometry, bent molecular geometry ($\theta \approx 109.5^\circ$)

\item \textbf{FIRE minimization (VSEPR mode):} Minimize with WCA potential ($\varepsilon = 0.001$\,kcal/mol) to resolve steric clashes. Result: O--H bonds at $\sim 0.95$\,\AA, H--O--H angle $\sim 105^\circ$

\item \textbf{MD exploration (optional):} Run 1000 steps of Langevin dynamics at 300\,K to sample vibrations

\item \textbf{FIRE minimization (MD mode):} Minimize with full LJ potential + bonded terms. Final result: O--H = 0.96\,\AA, H--O--H = 104.3$^\circ$
\end{enumerate}

\subsection{Validation}

\begin{center}
\begin{tabular}{lcc}
\toprule
Property & Experimental & Formation Engine \\
\midrule
O--H bond length & 0.9584\,\AA & 0.96\,\AA \\
H--O--H angle & 104.45$^\circ$ & 104.3$^\circ$ \\
Dipole moment & 1.85\,D & (not computed) \\
\bottomrule
\end{tabular}
\end{center}

The agreement is excellent. The geometry \textbf{emerged} from the force field and VSEPR theory, not from a lookup table.

\subsection{What This Proves}

If the engine can reproduce water's geometry from first principles, it can (in principle) predict the geometry of \textit{any} molecule within the domain of validity of the force field. The limitation is the force field, not the methodology.

\section{Known Limitations and Future Work}

\subsection{VSEPR Limitations}

\begin{itemize}
\item No automatic lone pair counting (must be specified manually)
\item Single central atom optimization (multi-center requires sequential application)
\item Coordination number $> 6$ is unreliable
\item No resonance structures or aromaticity
\end{itemize}

\subsection{Basin Mapping: Not Fully Automated}

The tools exist (FIRE, stationarity gate, Kabsch alignment), but the workflow is not integrated. A user must:
\begin{enumerate}
\item Write a script to perturb structures
\item Run FIRE on each perturbation
\item Collect energies and cluster by RMSD
\end{enumerate}

Automation is a future extension.

\subsection{Classification: Composition-Based Only}

The classifier uses only elemental composition, not 3D geometry. This means:
\begin{itemize}
\item Cannot distinguish stereoisomers (e.g., R vs. S enantiomers)
\item Cannot detect functional groups (e.g., alcohol vs. ether)
\item Cannot identify specific motifs (e.g., benzene ring vs. cyclohexane)
\end{itemize}

Geometry-based classification (using graph neural networks or structural fingerprints) is a future extension.

\subsection{Scoring: No Experimental Validation}

The scoring parameters ($\mu_1 = 8$, $\alpha = 0.3$, etc.) were chosen by intuition and manual tuning, not by fitting to a database of "interesting" vs. "boring" molecules. Validation against expert chemical judgment is pending.

\section{Conclusion: Formation as Prediction, Not Lookup}

This section has documented the formation physics framework \textbf{as it exists in the codebase}:

\begin{itemize}
\item \textbf{VSEPR geometry prediction:} Fully implemented in \texttt{src/pot/energy\_vsepr.hpp}
\item \textbf{MD + FIRE pipeline:} Documented in Section~5, produces formations via exploration + quenching
\item \textbf{Classification system:} 7 categories with explicit bonuses, implemented in \texttt{tools/classification.py}
\item \textbf{Scoring system:} 6-factor multiplicative formula, implemented in \texttt{tools/scoring.py}
\item \textbf{Basin mapping:} Conceptual framework defined, not fully automated
\end{itemize}

The central thesis: \textbf{formations are predictions, not lookups}. The engine does not store molecular geometries. It computes them from elemental properties and physical laws. If the force field is correct, the geometry must emerge. If it does not emerge, the force field has a defect.

This is the difference between a \textbf{database} and a \textbf{simulator}. The formation engine is the latter.

\paragraph{Transition to \S7:} The formation pipeline produces structures and energies. Section~7 documents the \textit{statistical interpretation layer} that converts these raw numbers into scientific conclusions: convergence detection, structural comparison via Kabsch alignment, and the observable tracking framework.

\end{document}
